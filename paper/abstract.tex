\begin{abstract}
This seminar paper deals extensively with the critical path method developed by
Morgan R. Walker and James E. Kelly in the late 1950s\cite{Kelley:1959:CPS:1460299.1460318}.
Despite the age of this
method, it is still widely used in a lot of different projects.  This paper is
divided into five chapters, each dealing with an important part of the critical
path method. 

Chapter one provides a short introduction into the topic, some background
information about the critical path method and definitions of some very
important catchwords. Furthermore the history of the critical path method is
outlined. 

In the subsequent chapter, the critical path method itself is described very
detailed, it is explained how information gained with the critical path method
can be interpreted and how to react on this information in a proper way (if
necessary).

Chapter three deals with different algorithms used to determine the critical
path of a project.  The mentioned algorithms appear in different tools, which
are used to determine the critical path of a project. 

In the fourth chapter, advantages and disadvantages of the critical path method
are outlined. Furthermore possible alternatives and advancements are mentioned
and described. 

Last but not least chapter five explains how the critical path method is used in
everyday project management, which project management tools support the critical
path method and a short introduction to this tools.

\end{abstract}
