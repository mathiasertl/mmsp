\section{Introduction}
% Background, Scope (5 Seiten) - Silvio

This first chapter provides a short introduction into the \cpm{} and some noteworthy
background information.

First of all it should be mentioned, that there exist a lot (100+) different names for  the critical
path method like "network scheduling", "critical path analysis" or "\cpm{}". According
to \cite{dua} there is no basic difference between this differently named methods. For the readers
convenience, in this seminar paper the method is always called \cpm{} (CPM). 

The \cpm{} is a widely used project management tool, which allows the scheduling of
activities in a project. It is not only used in software development but even in construction
projects, production projects, research projects, and a lot more. The \cpm{} can be
used for every project, consisting of different, interdependent activities.  

The purpose of the \cpm{} is to calculate\cite{santiago}:
\begin{itemize}
  \item The longest path of planned activities to the end of the project.
  \item The earliest and latest that each activity can start and finish without making the project longer.
\end{itemize}

A quote from an unknown reviewer of \cite{obrien} describes the \cpm{} very well:

\begin{quotation}
Perhaps the most ironic aspect of the \cpm{} is that after you understand it, it is
self-evident. Just as an algebra student can apply the rules without full appreciation of the power
of the mathematical concepts, so can the individual apply CPM or its equivalent without fully
appreciating the applicability of the method.
\end{quotation}

Activities on this critical path should be monitored very closely, and it is important to make sure
that such activities are completed in time. 

The \cpm{} is divided into 3 phases:
\begin{description}
  \item[Phase 1] Break project into activities.
  \item[Phase 2] Create time estimates for each activity. 
  \item[Phase 3] Create a time-cost relationship.
\end{description}
Each phase is described in more detail in chapter 2.

Generally, the CPM is a very powerful method to receive important information
about a project. It shows which activities in a project are most important, which activities can be
carried out in parallel, how much resources are needed, when the project can be finished, when it
will be finished according to the current state and it is possible to get an overview of all
activities which have to be carried out.

\subsection{\cpm{} - what for?}

Basically the \cpm{} is part of a scheduling technique as everyone uses it every day – just a bit more
advanced. For a single person scheduling of it’s everyday activities is rather easy (taking a
shower, dressing up, \ldots), but can get quite challenging if the person is stressed or if more
tasks
should be done at the same time. If it is necessary to schedule the tasks of two persons this is
even more challenging, for three or more persons it is nearly impossible without a well-defined,
structured method. Mapping all this to an industry project, a lot of different tasks have to be
managed and each task must be finished at a specific time. CPM allows saving time through better
planning and as everywhere in the industry time is money.

In December 1957 an experiment was started to show the efficiency of the \cpm{}. Two
teams consisting of different engineers were set up, one team had to use an early version of CPM (at
this time called Kelly-Walker method) the other team used traditional planning methods. Both groups
had to plan a ten million US dollar chemical plant in Louisville, Kentucky. Of course both team worked
completely independent of each other. The members of the CPM group received and introduction course
(40h) to CPM. All in all, the project contained a network of more then 800 activities. After six
month, major changes (about 40\% of the whole project) had to be taken into account. For the CPM team
the modification and recomputation of the plan took only about 10\% of the original planning time -
much better then the normal scheduling group.  Another major advantage of the CPM team was the
ability to identify critical delivery items. The CPM team identified this critical delivery items
from the analysis of their network plan, while the normal scheduling group selected critical 
delivery items randomly\cite{obrien}. 

This example shows that by using CPM a lot of time and effort can be spared. 

\subsection{History of the CPM}

Research in the field of project management and activity scheduling was done because of the need of
a method to plan and control complicated projects. 

As previously mentioned the \cpm{} was invented in the year 1959 by Morgan R. Walker
(DuPont) and James E. Kelly (Remington Rand).  Some other organizations claim that they invented the
\cpm{} (UK Central Electricity Generating Board and the US Navy) but officially the
chemical company DuPont invented it. The \cpm{} was based on a graphic network
commonly known as the “arrow method”.  The advantages of the \cpm{} method were discovered by
many organizations especially construction companies who quickly integrated the CPM in their daily
project management. 

Not only in the private sector, but even in the military sector such a method was required.
Therefore the US Navy started the project “Program Evaluation Research Task” codename PERT. Its goal
was to develop a way to “plan and control complicated projects”. According to the phase II report of
the PERT program, the method needs to provide following possibilities.

\begin{enumerate}
  \item Increased orderliness and consistency in planning and evaluating.
  \item An automatic mechanism for identifying potential trouble spots.
  \item Operational flexibility for a program by allowing for a simulation    of schedules.
  \item Rapid handling and analysis of integrated data to permit expeditious corrections.
\end{enumerate}

An outcome of this project was the first arrow diagram (developed by Dr. C.E. Clark). The PERT
project was later renamed in “Program Evaluation and Review Technique” and used in multiple large
scale projects like the “Fleet Ballistic Missle Program” with over 3000 parties involved. It is
said, that Morgen Walker and James Kelly took some ideas and the term "critical path" from this PERT
project\cite[p. 10]{obrien}.

In the year 1961 an advanced version of the CPM was presented by Professor Fondahl of the Stanford
University – this version was called “precedence method”. The precedence method was an improvement
in respect of schedule construction and analysis. Additionally, after this update by Prof. Fondahl,
the \cpm{} even was able to assist in the organization of resources. This is the
method, which is currently used in most projects\cite{Uher}.

The probably best known project which made use of the \cpm{} is the Apollo 11 mission
of the NASA. During this mission two persons were landed on the moon (Neil Armstrong, Buzz Aldrin)
on July 20, 1969. About two million tasks led to the moon landing. 

\subsection{Important terms}

In this part some important terms are introduced and described as detailed as needed for this
seminar paper. 

\begin{description}
  \item[Activity:] An \emph{activity} is a distinct activity within a project. It is a seperate unit of work
    which has a defined outcome (success condidtion) and possibly depends on the completion of other
    jobs.
  \item[Float (slack):] The \emph{float} is the amount of time an activity can be delayed without
    affecting succeeding tasks (free float) or the completion date of the project (total float).
    When using CPM in software development, the float is mostly called buffer. 
  \item[Critical path:] The \emph{critical path} is a series of activities whose combined duration is the minimal
    amount of time needed to complete the project.  If an activity on the critical path is delayed,
    the whole project is affected and delayed. A project can have several, parallel critical paths.
  \item[Critical activity:] Is an activity with zero buffer, therefore an activity on the critical
    path. An activity is defined as critical if the maximum time available is equal to the duration of the job\cite[p.
        163]{Kelley:1959:CPS:1460299.1460318}
  \item[floater:] A \emph{floater} is an activity that is \emph{not} part of any critical path. Thus, a delay in the
    completion of a floater does not influence the completion of the project if the delay is within
    certain defined limits. Floaters thus denote jobs can be rescheduled with relative ease without impacting the entire project.
  \item[Early finish date (EF):] The \emph{Early finish date} or EF is the earliest possible point 
    in time on which the project (or an activity) can be finished. The EF needs to be changed, if
    activities on the critical path are delayed. 
  \item[Early start date (ES):] The \emph{Early start date} or ES is the earliest possible point in
    time on which the project (or an activity) can start. 
  \item[Late finish date (LF):] The \emph{Late finish date} or LF is the latest possible point in
    time on which the project (or an activity) must be finished, in order not to delay a specific
    milestone or a fixed handover date.
  \item[Latest start date (LS):] The \emph{Latest start date} or LS is the latest possible point in
    time on which the project (or an activity) must start (based on previous effort estimation) in
    order not to delay a specific milestone or a fixed handover date.
\end{description}

\subsection{Summary}
Chapter one provides a short introduction into the \cpm{}. First of all it is
outlined, that the \cpm{} is a project management tool, which can be used in nearly
every kind of project, which consists of interdependent activities. Afterwards it is described which
information about the project can be obtained with the help of the \cpm{} and how to
handle this information.  

After the introduction an experiment of the year 1959 is presented. This experiments shows the
advantages of the \cpm{} compared to traditional methods. 

The third part of chapter one is a look into the past of the \cpm{}. It is explained
that Morgan Walker and James Kelly developed this method; it is even mentioned why they created this
method and which other approaches exist.  

The last part of chapter one is an introduction of important terms in respect of the \cpm{}.
The two most important terms are float (how much a task can be delayed) and of course
critical path (shortest path to the end of the project)\cite{santiago}.
