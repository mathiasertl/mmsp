\section{Advantages and disadvantages of the \cpm{}}

This chapter provides an overview over advantages and disadvantages of the critical path method.

\subsection{Advantages of the \cpm{}}

The following list shows some of the reasons why the critical path method is used.

\begin{itemize}
\item The critical path method forces all members in the project team to identify all sorts of different
tasks that had to be completed for the project to finish. That process again forces the members to
identify all the requirements for the different tasks in a logical and critical way. Regardless of
whether an activity is a successor or predecessor of another, they all need their own analysis and
evaluation. This becomes very important if tasks proceed on different locations and the cost and
time element is also dependent on external factors that have the capability to affect the whole
project time.

\item A very important fact from analyzing the network diagram is to identify the critical path for the
project. Provided this information, the management of the project has an appropriate assessment of
the possible problems that could occur and the associated tasks. Often, the critical path decides
the allocation of resources and by the interpretation of the network diagram it is ensured that a
resource is not allocated to more than one task for a specified period of time.

\item The network diagram created while using the CPM provides an outlook of the completion time of the
project and supports the responsible persons in the scheduling and planning of the tasks that have
to be done for finishing the project.

\item CPM also supports a logical and disciplined approach to scheduling and managing a project over a
long period of time. Finding the factors that can earnestly affect the project is very difficult and
not finding them is the main cause of project overruns. By forcing the members in a project team to
identify all the tasks, attention to details can be achieved and consequently a precise view of the
processes and the time and cost associated with them is obtained.

\item It is possible with CPM to optimize the time-cost relationship as you can visually identify the
tasks which can lead to problems if not monitored and managed effectively over a specific period of
time. Cost structures often differ between the normal organization and projects. Identifying the
exact cost of a project is not trivial and is not universal to all companies not even to all
projects within a company. Developing time-cost relationships for projects requires that the
responsible persons are able to identify the main cause of the problems that are affecting the time
and cost variable.

\item When the time-cost variables are known, the project can be optimized to fulfill the aims and goals
of the company. For example, is a project team is able to identify that they need more resources if
the project has to be within a certain time or vice versa this fact is clear right from the start of
the project. But it is impossible in the initial stage to know all the factors that can influence a
task, but the majority of factors and variables can be understood and so the risks and uncertainties
are reduced.

\item Cause critical paths do not remain static and change during the life time of a project due to
external or internal factors it is important to track the CPM. In this way the managers can
recognize areas where attention needs to be focused.

\item CPM identifies the entire chain of tasks and so supports the scheduling of them. Often, at the
beginning of a project the cost requirements and number of tasks are high. But with advancing
project state the tasks might sort themselves out into routine or critical. Managers, instead of
focusing on the entire issue, can focus on task groups that are immediate and can impact the
following tasks.

\item With CPM managers can identify the slack and float time of a project. Knowing them helps the
managers to reallocate resources to different tasks or supports them at shifting and moving of tasks
to optimize the utilization of resources.

\item Critical paths are also updated periodically for any project and offer the project manager and
members a visual representation of the completion of various stages of the project and easily
identify problem areas where further attention might be required.

\item Often, projects have more than one critical path and when this situation arises, CPM supports
managers on creating a plan of actions to handle the multiple critical paths.

\item CPM can also help estimate the project duration and this information can be used to minimize the sum
of direct and indirect costs involved in the project planning and scheduling.

\item The knowledge of the location of critical activities is vital for the effective planning and control
of a project. It assists the planner in allocating sufficient resources to critical activities
to ensure their completion on time. Furthermore CPM can identify the paths that can be taken to
accelerate a project to be completed prior to its due date or identify the shortest possible time or
the least possible cost that is needed to complete a task.

\item A very good reason to use CPM is that it offers the company a sort of documentation that can be
reused for similar projects in the future. Documenting tasks and the main causes of the problems
occurred while executing them can help to avoid doing the same mistakes in the future. But also the
documentation of time consumption and cost factors can be used in future projects to guess these
factors more exactly.

\item CPM methods are based on deterministic models and the estimation of time tasks are based on
historical data maintained within the organization or data obtained from external sources.
\end{itemize}

\subsection{Disadvantages of the \cpm{}}

The following list shows the main disadvantages of the critical path method. Many of them are as a
result of the technical and conceptual factors involved in the CP analysis (CPA) process.

\begin{itemize}
\item Many interconnections between the tasks can result in the network diagram to become very complicated
and so the CPA process becomes more complicated too. Consequently the risk of making failures on
calculating the critical path becomes higher.

\item It is very important that the project members are familiar with the different tasks that have to be
done so they can make realistic estimates on time consumption and cost with given resources. If the
project is unique and has never been undertaken by the company in the past, there is no historical
data to orient and time consumption and cost requirements have to be speculated.

\item If there is more than one critical path it gets more complicated to understand the requirements (in
view of resources) for the paths. It can even get worse if the critical paths take place parallel.

\item As mentioned as an advantage the changing of critical paths is, of course, a big disadvantage.
Critical paths can become invalid and new critical paths have to be identified. Consequently the
managers and project members have to constantly review the network diagram to identify the movement
and shifting of the critical path over time.

\item With changing critical paths and floats also the scheduling of personnel changes. Reallocation of
personnel is by far not trivial, especially if the individual is working on more than one project
(probably even more than one critical path) at a time.

\item CPA and network diagrams are highly dependent of information technology and computer software. The
initial set up costs for such systems can be very high and maintaining them can also become
expensive if the company does not have in house capabilities.

\item Although CPM is very valuable in the extent of details that it provides, modifying the system
constantly can be difficult especially if it is associated with reallocation of resources and time.

\item CPM highly depends on the efficiency of the communication networks within a company. Without good
communication it is not possible to get the collected data from previous projects. Without this data
important factors have to be estimated which is not the intent of CPM.

\item Sometimes projects use different calendars for the scheduling and planning and this can cause more
complications. Scheduling a project using the combination of different calendars can create
confusion in the CPM if the individual analyzing the CP is not careful about evaluating the type of
calendar used for different tasks in the network diagram.

\item Many projects have a long duration (3-5 years). In this time the personnel involved in a project
often changes. Some left the company, some get transferred to other departments and some are
retired. The new member might not be as well versed with the initial concepts and brainstorming that
went into the creation of the network diagram. Changes and modifications made over the period of
time on the network diagram can also be difficult to track if a good method of documentation of the
changes is not made. Poor documentation is often the reason for repeated mistakes.

\item CPA does not take into account the learning curve for new members on the project or for tasks that
are new and unique to the project. Using past information of learning curves can help project
managers but CPM does not traditionally consider this as an important variable for allocation of
time or resources. 
\end{itemize}
%(Peter & Roy, 2009)
%Peter, S., & Roy, G. L. (2009). Projects’ Analysis through CPM (Critical Path Method).
