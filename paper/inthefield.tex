\section{Critical Path Method in practice}
Primary a network plan delivers a time economic planning of a sequence of tasks. In addition , it also provides an excellent mean for flow control. For example you have the possibility to check whether a task is solved or it is open. Moreover you can avoid critical path. Therefore, a Critical Path Method network plan is a good opportunity to manage projects. If you think about a project then two kinds of projects come into your mind. On the one hand small projects, which can be handled manually and on the other hand there are big projects, which would be a real challenge to handle manually. Therefore it is necessary to use software solutions, that make the work more convenient and efficient. 
For this reason this chapter would focus on “Critical Path Method” in practice. At first we would focus on simple Critical Path Method calculation which can be done with simple spreadsheet applications. We will demonstrate this in a small OpenOffice example. Moreover this is a good approach to illustrate how “Critical Path Method” works.
In a further step we would focus on “real” Critical Path Method tools. On the one hand we would give an overview over MS Project, which is the Microsoft's commercial approach of a project planning software. And we would also demonstrate “OpenProj” which is an open source approach of an project management application. Afterwards we would compare them and give a listing with a lot of other useful project management software. 
Finally we would finish this chapter with a conclusion part.

