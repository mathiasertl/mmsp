\section{The Critical Path Method}
%Eigentliche Beschreibung der Methode (5 Seiten) - Mathias
% overview:
%http://hspm.sph.sc.edu/COURSES/J716/CPM/CPM.html

The \cpm{} (or just CPM) was originally introduced by James Kelley and Morgan Walker in
\cite{Kelley:1959:CPS:1460299.1460318}. The foundation of the \cpm{} is an abstract notion of a
\emph{project}. A project is assumed to consist of \emph{jobs} that represent individual activities
and are executed seperately from each other. Each job may depend on the completion of other jobs and
in turn must be completed for other jobs to begin. The central goal of the \cpm{} is to compute a list
of \emph{critical jobs}, where a delay in the completion of a job delays the entire project.

The \cpm{} seperates planning from scheduling. In the planning phase, a planner defines all
activities and the order in which they must take place. The planner must also consider required 
technology, resources, people, etc. involved. A scheduler can then use the defined activities to
devise a schedule based on the plan and on costs\cite[p. 161]{Kelley:1959:CPS:1460299.1460318}.

\cpm{} has evolved since its original incarnation to include other aspects essential to the
successfull completion of a project, such as required resources and resource-leveling. Various tools
and techniques have been developed to optimize and shorten the critical path.

\subsection{Terminology}
The \cpm{} defines a few distinct terms which deserve a seperate explanation:
\begin{itemize}
  \item A \emph{job} is a distinct activity within a project. It is a seperate unit of work which
  has a defined outcome (success condidtion) and possibly depends on the completion of other jobs.
  \item A \emph{critical path} defines a list of jobs depending on each other where a delay in the
  completion of one job will delay the entire project. A project might have multiple critical paths.
  \item A \emph{critical job} is a job that is part of one or more critical paths. A job is defined
  as critical if the maximum time available is equal to the duration of the job\cite[p.
  163]{Kelley:1959:CPS:1460299.1460318}.
  \item A \emph{floater} is a job that is not part of any critical path. Thus, a delay in the
  completion of a floater does not influence the completion of the project right away. Floaters thus
  denote jobs can be rescheduled with relative ease without impacting the entire project.
\end{itemize}

\subsection{What you can achieve using the \cpm}
The seperation of planning and scheduling allows the project management to dynamically reschedule
parts of the project during its execution if real-world events force a deviation from the plan.
Reasons for deviation may be a natural disaster, a delay in the shipment of required resources or
a late completion of a critical job. Practical experience shows, however, that rescheduling is a
complex and error-prone task, because i.e. moving job A to next week might mean that job B cannot be
executed because required resources are not available. %TODO: Da gibts ne quelle irgendwo dafuer.
\subsection{How to apply the \cpm}
* draw up list of jobs
* make Activity or Network diagram
* apply critical path method to decide: "How long will a project take"
\subsection{Common errors}
\subsection{Possible improvements}
more information, to shorten project time, assign resources, etc.

\subsection{Alternatives}
