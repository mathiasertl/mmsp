\section{The Critical Path Method}
%Eigentliche Beschreibung der Methode (5 Seiten) - Mathias
% overview:
%http://hspm.sph.sc.edu/COURSES/J716/CPM/CPM.html

The \cpm{} (or just CPM) was originally introduced by James Kelley and Morgan Walker in
\cite{Kelley:1959:CPS:1460299.1460318}. The foundation of the \cpm{} is an abstract notion of a
\emph{project}. A project is assumed to consist of \emph{jobs} that represent individual activities
and are executed seperately from each other. Each job may depend on the completion of other jobs and
in turn must be completed for other jobs to begin. The central goal of the \cpm{} is to compute a list
of \emph{critical jobs}, where a delay in the completion of a job delays the entire project.

The \cpm{} seperates planning from scheduling. In the planning phase, a planner defines all
activities and the order in which they must take place. The planner must also consider required 
technology, resources, people, etc. involved. A scheduler can then use the defined activities to
devise a schedule based on the plan and on costs\cite[p. 161]{Kelley:1959:CPS:1460299.1460318}.

\cpm{} has evolved since its original incarnation to include other aspects essential to the
successfull completion of a project, such as required resources and resource-leveling. Various tools
and techniques have been developed to optimize and shorten the critical path.

\subsection{What you can achieve using the \cpm}
The seperation of planning and scheduling allows the project management to dynamically reschedule
parts of the project during its execution if real-world events force a deviation from the plan.
Reasons for deviation may be a natural disaster, a delay in the shipment of required resources or
a late completion of a critical job. Practical experience shows, however, that rescheduling is a
complex and error-prone task. Moving a job may be simple in theory, but might in fact be very
complex. For example, moving a job to next week might mean that a required specialist is not
available for a different job that is scheduled during that time period.
%TODO: Da gibts ne quelle irgendwo dafuer.

\subsection{Initial planning}\label{subsec:initial_planning}
The first thing when applying the \cpm{} is to draw up a list of activities. This has to take place
\emph{before} any real work is done on the project and might be considered an initial kick-off of
the project. This list can be done in the
form of a \emph{work-breakdown structure} or in any more or less formal structure. The planning team
has to take great care that this stage is done correctly, as any errors could cause big problems
during the execution of the plan.

When the planning team decides on an activity, some information on it should be compiled:
\begin{enumerate}
  \item A distinctive, descriptive and unique name. 
  \item An estimated total time that the activity will take to complete. Depending on the
    requirements, other estimates (minimum time, maximum time, \ldots) might also be of interest.
  \item Activities that must be completed before this respective activity. It is also good practice
    to (informally, at first) write down other activities that depend on the activity.
\end{enumerate}

One important question that should be considered is the level of abstraction the plan should have.
For example, when trying to build a large oil-tanker, it doesn't really make sense to have the
activities "build outer hull" and "tighten screws of the captains chair" on the same plan. The
chosen abstraction of the plan depends on
\begin{itemize}
  \item Who creates the list of activities. The list of activities might even be drawn up by higher
    levels of management and only then handed down to different devisions of the planning
    departement. As such, the activities might already indicate a certain division of work. 
  \item Who will read and work with the plan. A plan based on the \cpm{} might also be used to 
    formally communicate i.e. between different management divisions, autonomous sections of a 
    company or other companies alltogether. Depending on the recipient of the plan and his
    experience and knowledge, the a different level of detail might become necessary.
  \item The overall complexity of the project. For a very large and complex project comprised of
    many different individual activites, a lower level of abstraction might not be practical. A
    possible solution is to logically split the project into individual parts, so that each part
    becomes seperated (albeit depending on each other) from each other. 
  \item Dependency on other activities. A defining axiom of the \cpm{} is that each activity can
    only start when all activities it depends on are fully completed\cite[p. 37]{obrien}. Following
    this axiom, it might become necessary to split an activity A into two parts, if activity B
    depends on the first part of A and the second part of A depends on activity B.
\end{itemize}

One common mistake made in the implementation of the \cpm{} is to assume that software support tools
(see section~\ref{sec:inthefield}) either draw up a list of activities for you or make drawing up
such a list a lot easier. Software support tools do not (and cannot) aim to ease this task, they can
only process the information they receive. If this initial work is not done carefully, any diagrams
etc. drawn by software are not any better than if no software were involved\cite[p. 39]{obrien}.

\subsection{Drawing the activity diagram}
An \emph{activity diagram}, often also called \emph{network diagram}, is the central part of the
\cpm{}. The diagram is created based upon the list of activities devised in
section~\ref{subsec:initial_planning}. 

The most important element of an activity diagram is of course the activity, represented by a simple
arrow. Depending on the complexity of the graph, arrows might be descriptivly labeld or labeld with
references to the actual activity that may be looked up in a reference table. Such a table is needed
in any case to store additional information (such as duration, etc.). 

When an activity is completed, an \emph{event} is said to occur. Such an event has no 
duration but represents an instance in time. 

\subsection{Find critical paths}
* apply critical path method to decide: "How long will a project take"

\subsection{Common errors}

\subsection{Possible improvements}
more information, to shorten project time, assign resources, etc.

\subsection{Alternatives}
PERT?
